% Options for packages loaded elsewhere
\PassOptionsToPackage{unicode}{hyperref}
\PassOptionsToPackage{hyphens}{url}
\PassOptionsToPackage{dvipsnames,svgnames,x11names}{xcolor}
%
\documentclass[
  number,
  review]{elsarticle}

\usepackage{amsmath,amssymb}
\usepackage{iftex}
\ifPDFTeX
  \usepackage[T1]{fontenc}
  \usepackage[utf8]{inputenc}
  \usepackage{textcomp} % provide euro and other symbols
\else % if luatex or xetex
  \usepackage{unicode-math}
  \defaultfontfeatures{Scale=MatchLowercase}
  \defaultfontfeatures[\rmfamily]{Ligatures=TeX,Scale=1}
\fi
\usepackage{lmodern}
\ifPDFTeX\else  
    % xetex/luatex font selection
\fi
% Use upquote if available, for straight quotes in verbatim environments
\IfFileExists{upquote.sty}{\usepackage{upquote}}{}
\IfFileExists{microtype.sty}{% use microtype if available
  \usepackage[]{microtype}
  \UseMicrotypeSet[protrusion]{basicmath} % disable protrusion for tt fonts
}{}
\makeatletter
\@ifundefined{KOMAClassName}{% if non-KOMA class
  \IfFileExists{parskip.sty}{%
    \usepackage{parskip}
  }{% else
    \setlength{\parindent}{0pt}
    \setlength{\parskip}{6pt plus 2pt minus 1pt}}
}{% if KOMA class
  \KOMAoptions{parskip=half}}
\makeatother
\usepackage{xcolor}
\setlength{\emergencystretch}{3em} % prevent overfull lines
\setcounter{secnumdepth}{5}
% Make \paragraph and \subparagraph free-standing
\makeatletter
\ifx\paragraph\undefined\else
  \let\oldparagraph\paragraph
  \renewcommand{\paragraph}{
    \@ifstar
      \xxxParagraphStar
      \xxxParagraphNoStar
  }
  \newcommand{\xxxParagraphStar}[1]{\oldparagraph*{#1}\mbox{}}
  \newcommand{\xxxParagraphNoStar}[1]{\oldparagraph{#1}\mbox{}}
\fi
\ifx\subparagraph\undefined\else
  \let\oldsubparagraph\subparagraph
  \renewcommand{\subparagraph}{
    \@ifstar
      \xxxSubParagraphStar
      \xxxSubParagraphNoStar
  }
  \newcommand{\xxxSubParagraphStar}[1]{\oldsubparagraph*{#1}\mbox{}}
  \newcommand{\xxxSubParagraphNoStar}[1]{\oldsubparagraph{#1}\mbox{}}
\fi
\makeatother


\providecommand{\tightlist}{%
  \setlength{\itemsep}{0pt}\setlength{\parskip}{0pt}}\usepackage{longtable,booktabs,array}
\usepackage{calc} % for calculating minipage widths
% Correct order of tables after \paragraph or \subparagraph
\usepackage{etoolbox}
\makeatletter
\patchcmd\longtable{\par}{\if@noskipsec\mbox{}\fi\par}{}{}
\makeatother
% Allow footnotes in longtable head/foot
\IfFileExists{footnotehyper.sty}{\usepackage{footnotehyper}}{\usepackage{footnote}}
\makesavenoteenv{longtable}
\usepackage{graphicx}
\makeatletter
\newsavebox\pandoc@box
\newcommand*\pandocbounded[1]{% scales image to fit in text height/width
  \sbox\pandoc@box{#1}%
  \Gscale@div\@tempa{\textheight}{\dimexpr\ht\pandoc@box+\dp\pandoc@box\relax}%
  \Gscale@div\@tempb{\linewidth}{\wd\pandoc@box}%
  \ifdim\@tempb\p@<\@tempa\p@\let\@tempa\@tempb\fi% select the smaller of both
  \ifdim\@tempa\p@<\p@\scalebox{\@tempa}{\usebox\pandoc@box}%
  \else\usebox{\pandoc@box}%
  \fi%
}
% Set default figure placement to htbp
\def\fps@figure{htbp}
\makeatother
% definitions for citeproc citations
\NewDocumentCommand\citeproctext{}{}
\NewDocumentCommand\citeproc{mm}{%
  \begingroup\def\citeproctext{#2}\cite{#1}\endgroup}
\makeatletter
 % allow citations to break across lines
 \let\@cite@ofmt\@firstofone
 % avoid brackets around text for \cite:
 \def\@biblabel#1{}
 \def\@cite#1#2{{#1\if@tempswa , #2\fi}}
\makeatother
\newlength{\cslhangindent}
\setlength{\cslhangindent}{1.5em}
\newlength{\csllabelwidth}
\setlength{\csllabelwidth}{3em}
\newenvironment{CSLReferences}[2] % #1 hanging-indent, #2 entry-spacing
 {\begin{list}{}{%
  \setlength{\itemindent}{0pt}
  \setlength{\leftmargin}{0pt}
  \setlength{\parsep}{0pt}
  % turn on hanging indent if param 1 is 1
  \ifodd #1
   \setlength{\leftmargin}{\cslhangindent}
   \setlength{\itemindent}{-1\cslhangindent}
  \fi
  % set entry spacing
  \setlength{\itemsep}{#2\baselineskip}}}
 {\end{list}}
\usepackage{calc}
\newcommand{\CSLBlock}[1]{\hfill\break\parbox[t]{\linewidth}{\strut\ignorespaces#1\strut}}
\newcommand{\CSLLeftMargin}[1]{\parbox[t]{\csllabelwidth}{\strut#1\strut}}
\newcommand{\CSLRightInline}[1]{\parbox[t]{\linewidth - \csllabelwidth}{\strut#1\strut}}
\newcommand{\CSLIndent}[1]{\hspace{\cslhangindent}#1}

\makeatletter
\@ifpackageloaded{caption}{}{\usepackage{caption}}
\AtBeginDocument{%
\ifdefined\contentsname
  \renewcommand*\contentsname{Table of contents}
\else
  \newcommand\contentsname{Table of contents}
\fi
\ifdefined\listfigurename
  \renewcommand*\listfigurename{List of Figures}
\else
  \newcommand\listfigurename{List of Figures}
\fi
\ifdefined\listtablename
  \renewcommand*\listtablename{List of Tables}
\else
  \newcommand\listtablename{List of Tables}
\fi
\ifdefined\figurename
  \renewcommand*\figurename{Figure}
\else
  \newcommand\figurename{Figure}
\fi
\ifdefined\tablename
  \renewcommand*\tablename{Table}
\else
  \newcommand\tablename{Table}
\fi
}
\@ifpackageloaded{float}{}{\usepackage{float}}
\floatstyle{ruled}
\@ifundefined{c@chapter}{\newfloat{codelisting}{h}{lop}}{\newfloat{codelisting}{h}{lop}[chapter]}
\floatname{codelisting}{Listing}
\newcommand*\listoflistings{\listof{codelisting}{List of Listings}}
\makeatother
\makeatletter
\makeatother
\makeatletter
\@ifpackageloaded{caption}{}{\usepackage{caption}}
\@ifpackageloaded{subcaption}{}{\usepackage{subcaption}}
\makeatother
\journal{Open Quaternary}

\usepackage{bookmark}

\IfFileExists{xurl.sty}{\usepackage{xurl}}{} % add URL line breaks if available
\urlstyle{same} % disable monospaced font for URLs
\hypersetup{
  pdftitle={Biogeography of crop progenitors and wild plant resources during the transition to agriculture in West Asia, 21--8.3 ka},
  pdfauthor={Joe Roe; Amaia Arranz-Otaegui},
  colorlinks=true,
  linkcolor={blue},
  filecolor={Maroon},
  citecolor={Blue},
  urlcolor={Blue},
  pdfcreator={LaTeX via pandoc}}


\setlength{\parindent}{6pt}
\begin{document}

\begin{frontmatter}
\title{Biogeography of crop progenitors and wild plant resources during
the transition to agriculture in West Asia, 21--8.3 ka}
\author[1,2]{Joe Roe%
\corref{cor1}%
}
 \ead{joeroe@hey.com} 
\author[3]{Amaia Arranz-Otaegui%
%
}


\affiliation[1]{organization={University of
Bern},country={Switzerland},countrysep={,},postcodesep={}}
\affiliation[2]{organization={University of
Copenhagen},country={Denmark},countrysep={,},postcodesep={}}
\affiliation[3]{organization={University of the Basque
Country},country={Spain},countrysep={,},postcodesep={}}

\cortext[cor1]{Corresponding author}


        
\begin{abstract}
This paper presents the first comprehensive reconstructions of the
palaeodistributions of {[}X{]} plant species known to have been gathered
or cultivated by early agricultural societies in Southwest Asia,
including the progenitors of the first crops. We used machine learning
to train an ecological niche model (ENM) of each species based on its
present-day distribution in relation to climate and environmental
variables. Predictions of the potential ranges of these species at key
stages of the Pleistocene--Holocene transition could then be derived
from these models using hindcast data from palaeoclimate simulations.
Species ranges were on average {[}X\%{]} {[}larger\textbar smaller{]} in
the Early Holocene compared to today, indicating {[}\ldots{]}. The
modelled ranges predict the observed occurrence of these species on
archaeological sites with {[}low\textbar medium\textbar high{]}
accuracy. The regional ubiquity of species in the archaeological record
is {[}not{]} correlated with the predicted size of its range and the
diversity of archaeobotanical assemblages is {[}not{]} correlated with
the predicted diversity of its environs. This indicates that trends in
taxonomic composition of the archaeobotanical record is {[}not{]} likely
to be influenced by environmental change and species turnover, not just,
as is often assumed, human economic choices.
\end{abstract}





\end{frontmatter}
    

\begin{itemize}
\tightlist
\item[$\square$]
  Finish analysis:

  \begin{itemize}
  \tightlist
  \item[$\square$]
    Add terrain, soil
  \item[$\square$]
    Find a resolution that's workable
  \item[$\square$]
    Do thresholding?
  \end{itemize}
\item[$\square$]
  First draft:

  \begin{itemize}
  \tightlist
  \item[$\square$]
    Introduction
  \item[$\square$]
    Background

    \begin{itemize}
    \tightlist
    \item[$\square$]
      Biogeography
    \item[$\square$]
      ENM
    \end{itemize}
  \item[$\boxtimes$]
    Methods \& Materials

    \begin{itemize}
    \tightlist
    \item[$\boxtimes$]
      Occurrence data
    \item[$\boxtimes$]
      Predictor data
    \item[$\boxtimes$]
      Random Forest model
    \end{itemize}
  \item[$\square$]
    Results

    \begin{itemize}
    \tightlist
    \item[$\square$]
      Model assessment
    \item[$\square$]
      Hindcasting
    \end{itemize}
  \item[$\square$]
    Discussion

    \begin{itemize}
    \tightlist
    \item[$\square$]
      General trends
    \item[$\square$]
      Case studies
    \end{itemize}
  \item[$\square$]
    Conclusion
  \end{itemize}
\item[$\square$]
  Figures

  \begin{itemize}
  \tightlist
  \item[$\square$]
    tbl-predictors
  \item[$\square$]
    tbl-climate-periods
  \end{itemize}
\item[$\square$]
  Appendix with all hindcast predictions
\item[$\square$]
  References \& copyediting
\item[$\square$]
  Final proofread
\end{itemize}

\section{Introduction}\label{introduction}

\begin{itemize}
\tightlist
\item
  The first farming societies had an ecological context
\item
  Subsistence is understood (largely) through archaeobotany and
  zooarchaeology; ecological context from environmental archaeology,
  palynology, palaeoclimate records, etc.

  \begin{itemize}
  \tightlist
  \item
    But these have a variety of biases (human selection, taphonomy,
    etc.)
  \item
    And at the end of the day only represent specific places --
    interpolation to the entire region is not straightforward
  \item
    Overlap makes it difficult to see where human choices depart from
    the environmental background (cf. Martin et al., 2016)
  \item
    Or to fully contextualise subsistence strategies and shifts in
    strategies in response to environmental shifts (e.g. Yaworsky et
    al., 2023).
  \end{itemize}
\item
  Here we present an alternative approach using ENM

  \begin{itemize}
  \tightlist
  \item
    Whole region, at multiple climate snapshots
  \item
    Independent of archaeobot. and pal. clim. data, so can verify and
    compare
  \end{itemize}
\end{itemize}

\section{Background}\label{background}

\begin{itemize}
\tightlist
\item
  The transition to agriculture in West Asia was\ldots{}
\end{itemize}

\subsection{Biogeography and agricultural
origins}\label{biogeography-and-agricultural-origins}

\begin{itemize}
\tightlist
\item
  Has always been important in study of agricultural origins

  \begin{itemize}
  \tightlist
  \item
    Historically: Vavilov, Pumpelly \& Childe
  \item
    Genetic studies tell us origin points, but not ranges
  \end{itemize}
\item
  Important to e.g.

  \begin{itemize}
  \tightlist
  \item
    Distinguish environmental from potentially anthropogenic change
    (\textbf{MartinEtAl2017?}; \textbf{MartinEtAl2025?})
  \item
    Reconstruct sequences of domestication (\textbf{YeomansEtAl2017?})
  \end{itemize}
\item
  Epipal./Neo. plant-based economies were diverse

  \begin{itemize}
  \tightlist
  \item
    More than the ``founder crops'';
  \item
    More than food
  \item
    (In archaebot., not all intentionally collected)
  \item
    Regionally and temporally diverse
  \item
    \ldots so we model lots of species!
  \end{itemize}
\item
  Regional ecological reconstructions generally rely on the `expert
  interpolation' (or what do they call it with isoscapes?) method

  \begin{itemize}
  \tightlist
  \item
    See CSEAS (AEA-prep) presentation
  \item
    Figure: comparisons
  \end{itemize}
\end{itemize}

\subsection{Ecological niche modelling in
archaeology}\label{ecological-niche-modelling-in-archaeology}

Ecological niche modelling (ENM) or species distribution modelling (SDM)
is widely used by ecologists to predict the geographic range of a
species based on a set of environmental predictors. Essentially, it
involves combining records of where an organism has been observed with
environmental data (climate, topography, etc.) for those locations to
model the range of environmental values at which that species -- its
environmental niche. This model can then be used to predict the range of
the organism in question either in the same or a different environment.
(\textbf{CITE?}) suggests reserving the term `species distribution
modelling' for when the method is used to recover the verifiable range
of a species in a real and existing environment, and using `ecological
niche modelling' as the broader term covering hypothetical or predictive
applications -- a convention we follow here when referring to predictive
or `hindcast' models of past ranges. Within this overarching framework,
ecological niche modelling encompasses a wide range of applications and
a variety of potential environmental predictors, modelling approaches,
and methodologies, which we will not attempt to review here.

Ecological niche modelling has long been of interest to archaeologists
as both a means of exploring the biological niche of humans and for
reconstructing the past environments they inhabited (Franklin et al.,
2015; \textbf{DavidPollyEronen2011?}). In the first sense, it has been
used most extensively to model the range of humans and other hominin
species (e.g. \textbf{BenitoEtAl2017?}; \textbf{YousefiEtAl2020?};
\textbf{BanksEtAl2021?}; \textbf{YaworskyEtAl2024a?};
\textbf{YaworskyEtAl2024b?}; \textbf{GuranEtAl2024?}), especially in the
Palaeolithic. This overlaps with what archaeologists usually call
generically `predictive modelling'
(\textbf{VerhagenWhitley2020?})---more precisely `site distribution
modelling'---which is essentially the same approach as (and often
borrows methodologies from) ecological niche modelling but applied to
the occurrence of archaeological sites. Here what is modelled is not
strictly a biological niche alone, but also aspects of human geography,
taphonomy, and archaeological visibility. These applications can be
distinguihed from `palaeoecological niche modelling', where the object
of model remains, as in ecology, a non-human biological niche.

(Franklin et al., 2015) review palaeoecological niche modelling and
advocate for its greater adoption in environmental archaeology. One
relevant early example is (Conolly et al., 2012) used the occurrence of
wild and domestic \emph{Bos} remains at prehistoric archaeological sites
in Europe and West Asia to map the evolving niche of cattle over the
Pleistocene--Holocene transition. It has been used to model the
availability of fauna exploited by humans at wider scales (e.g.
\textbf{deAndresHerreroEtAl2018?}; \textbf{YaworksyEtAl2023?}) and, in a
West Asian context, of foraged plant resources in the landscape around
the Neolithic site of XX (Collins et al., 2018). Modelling the spread of
crops has been another significant archaeological application
(\textbf{CremaEtAl?}).

In the majority of studies to date (palaeo)ecological niche modelling
has been applied to archaeological data in an `inductive' fashion,
i.e.~faunal and botanical remains from ancient sites are used as the
occurrence dataset for training a model using either past or present
environmental data. However, both the zooarchaeological and
archaeobotanical records are sparse and subject to a complex array of
depositional, taphonomic and recovery biases factors that , many of
which are not fully understood and/or cannot be corrected for. This
means that while the archaeological attestation of the presence of a
species might generally be relied upon, it is highly unlikely that its
absence is representative of true past distributions.

The alternative approach is to train the model using contemporary
occurrence and environmental data and then use palaeoenvironmental data
to `hindcast' its predictions backwards in time. Like (Franklin et al.,
2015), we view the hindcasting approach as more promising, because
training datasets for both occurrences and environment are far more
readily available, complete and reliable for the present than the past.
There is some scepticism in the ecological niche modelling literature
about the ability of such models to make accurate predictions in unknown
environments (like the past) (\textbf{CITES?}), but here the hindcasting
approach also presents an opportunity: it reserves archaeological
occurrence data as an independent dataset that can be used to assess the
retrodictive performance of the model. This possibily was suggested by
(Franklin et al., 2015) but to our knowledge our study represents the
first attempt to actually do so.

The major practical limitation of the hindcasting approach is that it
relies on spatially explicit, high resolution palaeoenvironmental
surfaces with continuous coverage of the region and periods of interest.
Until recently, this has not been widely available for most
applications, which is perhaps why only a minority of studies use it
(cf. Yaworsky et al., 2023). In this study, we are able to take
advantage of the increasing availability of high resolution, global
palaeoclimate data derived from simulation experiments with general
circulation models of climate (Brown et al., 2018;
\textbf{BrownEtAl2020?}; \textbf{KargerEtAl2023?}).

\section{Methods and materials}\label{methods-and-materials}

\subsection{Occurrence data}\label{occurrence-data}

We consider X distinct taxa (Table~\ref{tbl-occ-count}) - all the
identifiable species known to be present at more than three Neolithic
sites in West Asia, according to our previous study (Arranz-Otaegui and
Roe, 2023).

Taxonomic names were resolves to the canonical form specified in the
GBIF Backbone Taxonomy (\textbf{GBIFSecretariat2023?}). So for example
occurrences for ``Bolboschoenus'' included all species and subspecies,
including specimens described as \emph{Bolboschoenus sp.},
\emph{Bolboschoenus maritimus} and \emph{Scirpus maritimus}. Domestic
species meeting our inclusion criteria were substituted with their wild
progenitor(s), where known.

\subsection{Occurrence data}\label{occurrence-data-1}

Occurrence data was obtained from the Global Biodiversity Information
Facility (GBIF) using via its application programming interface and the
R package rgbif (Chamberlain et al., 2024; Chamberlain and Boettiger,
2017). Although ENMs have reasonable predictive power even with small
training samples (Hernandez et al., 2006; Stockwell and Peterson, 2002;
Wisz et al., 2008), we excluded X taxa with less than fifty recorded
occurrences in our study region. We also excluded one taxon (\emph{Avena
sterilis}) with over 47,000 occurrences, as this would have been
computationally prohibitive and we were uncertain what account for such
a disproportionately high number of records.

Occurrence data only tells us where a species is present; there is
rarely definitive information on where the species is \emph{not} found.
We therefore need to generate random background points or
``pseudo-absences'' to feed to the model. There are several ways to do
this. We follow the advice of (Barbet-Massin et al., 2012) for
regression-based species distribution models and use a large (:10000)
random sample of points, weighted equally against the presences in the
regression. (Valavi et al., 2022) also recommend using a very large
background sample for random forest models.

\subsection{Predictor data}\label{predictor-data}

We modelled the occurrence of species as a function of X spatial
predictor variables (\textbf{?@tbl-predictors}). These included:

\begin{itemize}
\tightlist
\item
  Sixteen `bioclimatic' variables derived from monthly temperature and
  precipitation values, following standard practice for species
  distribution models (Hijmans et al., 2005). Contemporary bioclimatic
  predictor data for West Asia was extracted from the global CHELSA
  dataset (Karger et al., 2017), which predicts temperature and
  precipitation from downscaled general circulation model output at 1 km
  resolution.
\end{itemize}

\begin{itemize}
\tightlist
\item
  Terrain aspect and slope, which at high resolution perform well as
  proxies for solar radiation when modelling plant occurrence (Austin
  and Van Niel, 2011; Leempoel et al., 2015); and the topographic
  wetness index (TWI), which serves as a proxy for soil moisture and is
  particularly important in modelling arid environments (Campos et al.,
  2016; Di Virgilio et al., 2018; Kopecký and Čížková, 2010). All three
  were derived from the SRTM15+ digital elevation model using algorithms
  from WhiteboxTools (Lindsay, 2016).
\end{itemize}

\begin{itemize}
\tightlist
\item
  Edaphic data from SoilGrids (Hengl et al., 2017, 2014), which improves
  model performance for plants (Dubuis et al., 2013; Mod et al., 2016;
  Velazco et al., 2017). Based on a recent assessment of the reliability
  of SoilGrids data for species distribution modelling (Miller et al.,
  2024), we used a subset of four variables relating to soil texture
  (clay, silt, sand) and pH at the surface (0-5 cm depth).
\end{itemize}

Predictor data was transformed to the same projection system (WGS84 /
UTM 37 N) and a common resolution of X km.

For hindcasting, we used reconstructed bioclimatic data for key periods
(\textbf{?@tbl-climate-periods}) generated from downscaled paleoclimate
simulations from the HadCM3 general circulation model (Brown et al.,
2018). Terrain and soil predictors were held constant, since
reconstructions of these variables in the past are not available at
sufficient scale. It is not likely that either macroscale topography or
soil characteristics have altered significantly over the period of time
considered here, so we assume that this does not degrade model
performance, and may in fact benefit it by providing `anchoring'
predictors that are independent of climate change.

\subsection{Random Forest}\label{random-forest}

Ecological niche modelling is a classification problem that can be
approached with a wide range of statistical methods. A substantial
literature exists on the relatively performance of these approaches and
their respective parameterisations (reviewed in Valavi et al., 2022).
Random Forest, a widely-used machine learning algorithm, is amongst the
best performing approaches for presence-only species distribution
models, providing it is appropriately parameterised to account for the
class imbalance between presence and background samples (Valavi et al.,
2022, 2021). For our application, it also has the advantage of requiring
little to no manual parameter tuning to achieve good predictive results,
which makes it easier to model a larger numbers of taxa.

For each taxon we trained a classification model to predict occurrence
(presence or absence/background) based on our X predictor variables
(\textbf{?@tbl-predictors}). We used the Random Forest algorithm
implemented in the R package `ranger' (Wright and Ziegler, 2017) and the
`tidymodels' (\textbf{tidymodels?}) framework for data preprocessing and
model selection. To avoid overfitting, we follow (Valavi et al., 2021)
in their recommended hyperparameters and use of down-sampling to balance
presence and background samples. Models for each taxon were fit
independently, with redundant zero-variance predictors excluded, and
assessed based on balanced training (3/4) and test (1/4) partitions.

The output of the model is probabilistic. However, this should not be
understood as an actual probability of occurrence (\textbf{CITE?}), but
more akin to an estimate of habitat suitability. To simplify
interpretation, we can convert these predictions into binary
presence/absence maps, a process called ``thresholding''. We select the
threshold value for each model individually, using MaxSSS (as
recommended by Liu et al., 2013). This also makes it possible to
analyses the predictions together as an assemblage.

\section{Results}\label{results}

\subsection{Model assessment}\label{model-assessment}

\begin{itemize}
\tightlist
\item
  Modelled ecological niches on current data
\end{itemize}

\subsection{Hindcasting}\label{hindcasting}

Sensitivity to climate fluctuations?

\begin{itemize}
\tightlist
\item
  Comparison to archaeological occurrences
\end{itemize}

\section{Discussion}\label{discussion}

\begin{itemize}
\tightlist
\item
  General trends:

  \begin{itemize}
  \tightlist
  \item
    Quantified sensitivity of plant ranges to climate change
  \item
    Crop progenitors saw range contractions just before the onset of
    agriculture? (Moreso than other wild resources??)
  \item
    How could are reconstructed ranges at predicting archaeological
    assemblages? (+Implications)
  \end{itemize}
\item
  Interesting individual case studies:

  \begin{itemize}
  \tightlist
  \item
    Wheat progenitors
  \item
    ???
  \end{itemize}
\end{itemize}

\section{Conclusion}\label{conclusion}

\section*{References}\label{references}
\addcontentsline{toc}{section}{References}

\phantomsection\label{refs}
\begin{CSLReferences}{1}{0}
\bibitem[\citeproctext]{ref-ArranzOtaeguiRoe2023}
Arranz-Otaegui, A., Roe, J., 2023. Revisiting the concept of the
{``{Neolithic Founder Crops}''} in southwest {Asia}. Vegetation History
and Archaeobotany. \url{https://doi.org/10.1007/s00334-023-00917-1}

\bibitem[\citeproctext]{ref-AustinVanNiel2011}
Austin, M.P., Van Niel, K.P., 2011. Improving species distribution
models for climate change studies: Variable selection and scale. Journal
of Biogeography 38, 1--8.
\url{https://doi.org/10.1111/j.1365-2699.2010.02416.x}

\bibitem[\citeproctext]{ref-BarbetMassinEtAl2012}
Barbet-Massin, M., Jiguet, F., Albert, C.H., Thuiller, W., 2012.
Selecting pseudo-absences for species distribution models: How, where
and how many?: {How} to use pseudo-absences in niche modelling? Methods
Ecol. Evol. 3, 327--338.
\url{https://doi.org/10.1111/j.2041-210X.2011.00172.x}

\bibitem[\citeproctext]{ref-BrownEtAl2018}
Brown, J.L., Hill, D.J., Dolan, A.M., Carnaval, A.C., Haywood, A.M.,
2018. {PaleoClim}, high spatial resolution paleoclimate surfaces for
global land areas. Sci Data 5, 180254.
\url{https://doi.org/10.1038/sdata.2018.254}

\bibitem[\citeproctext]{ref-CamposEtAl2016}
Campos, V.E., Cappa, F.M., Viviana, F.M., Giannoni, S.M., 2016. Using
remotely sensed data to model suitable habitats for tree species in a
desert environment. Journal of Vegetation Science 27, 200--210.
\url{https://doi.org/10.1111/jvs.12328}

\bibitem[\citeproctext]{ref-rgbif}
Chamberlain, S., Barve, V., Mcglinn, D., Oldoni, D., Desmet, P.,
Geffert, L., Ram, K., 2024.
\href{https://CRAN.R-project.org/package=rgbif}{Rgbif: Interface to the
global biodiversity information facility API}.

\bibitem[\citeproctext]{ref-ChamberlainBoettiger2017}
Chamberlain, S., Boettiger, C., 2017.
\href{https://doi.org/10.7287/peerj.preprints.3304v1}{R python, and ruby
clients for GBIF species occurrence data}. PeerJ PrePrints.

\bibitem[\citeproctext]{ref-CollinsEtAl2018}
Collins, C., Asouti, E., Grove, M., Kabukcu, C., Bradley, L.,
Chiverrell, R., 2018. Understanding resource choice at the transition
from foraging to farming: {An} application of palaeodistribution
modelling to the {Neolithic} of the {Konya Plain}, south-central
{Anatolia}, {Turkey}. J. Archaeol. Sci. 96, 57--72.
\url{https://doi.org/10.1016/j.jas.2018.02.003}

\bibitem[\citeproctext]{ref-ConollyEtAl2012}
Conolly, J., Manning, K., Colledge, S., Dobney, K., Shennan, S., 2012.
Species distribution modelling of ancient cattle from early {Neolithic}
sites in {SW Asia} and {Europe}. Holocene 22, 997--1010.
\url{https://doi.org/10.1177/0959683612437871}

\bibitem[\citeproctext]{ref-DiVirgilioEtAl2018}
Di Virgilio, G., Wardell-Johnson, G.W., Robinson, T.P., Temple-Smith,
D., Hesford, J., 2018. Characterising fine-scale variation in plant
species richness and endemism across topographically complex, semi-arid
landscapes. Journal of Arid Environments 156, 59--68.
\url{https://doi.org/10.1016/j.jaridenv.2018.04.005}

\bibitem[\citeproctext]{ref-DubuisEtAl2013}
Dubuis, A., Giovanettina, S., Pellissier, L., Pottier, J., Vittoz, P.,
Guisan, A., 2013. Improving the prediction of plant species distribution
and community composition by adding edaphic to topo-climatic variables.
Journal of Vegetation Science 24, 593--606.
\url{https://doi.org/10.1111/jvs.12002}

\bibitem[\citeproctext]{ref-FranklinEtAl2015}
Franklin, J., Potts, A.J., Fisher, E.C., Cowling, R.M., Marean, C.W.,
2015. Paleodistribution modeling in archaeology and paleoanthropology.
Quat. Sci. Rev. 110, 1--14.
\url{https://doi.org/10.1016/j.quascirev.2014.12.015}

\bibitem[\citeproctext]{ref-HenglEtAl2014}
Hengl, T., de Jesus, J.M., MacMillan, R.A., Batjes, N.H., Heuvelink,
G.B.M., Ribeiro, E., Samuel-Rosa, A., Kempen, B., Leenaars, J.G.B.,
Walsh, M.G., Gonzalez, M.R., 2014. {SoilGrids1km--global} soil
information based on automated mapping. PLoS One 9, e105992.
\url{https://doi.org/10.1371/journal.pone.0105992}

\bibitem[\citeproctext]{ref-HenglEtAl2017}
Hengl, T., Mendes de Jesus, J., Heuvelink, G.B.M., Ruiperez Gonzalez,
M., Kilibarda, M., Blagotić, A., Shangguan, W., Wright, M.N., Geng, X.,
Bauer-Marschallinger, B., Guevara, M.A., Vargas, R., MacMillan, R.A.,
Batjes, N.H., Leenaars, J.G.B., Ribeiro, E., Wheeler, I., Mantel, S.,
Kempen, B., 2017. {SoilGrids250m}: {Global} gridded soil information
based on machine learning. PLoS One 12, e0169748.
\url{https://doi.org/10.1371/journal.pone.0169748}

\bibitem[\citeproctext]{ref-HernandezEtAl2006}
Hernandez, P.A., Graham, C.H., Master, L.L., Albert, D.L., 2006. The
effect of sample size and species characteristics on performance of
different species distribution modeling methods. Ecography 29, 773--785.
\url{https://doi.org/10.1111/j.0906-7590.2006.04700.x}

\bibitem[\citeproctext]{ref-HijmansEtAl2005}
Hijmans, R.J., Cameron, S.E., Parra, J.L., Jones, P.G., Jarvis, A.,
2005. Very high resolution interpolated climate surfaces for global land
areas. Int. J. Climatol. 25, 1965--1978.
\url{https://doi.org/10.1002/joc.1276}

\bibitem[\citeproctext]{ref-KargerEtAl2017}
Karger, D.N., Conrad, O., Böhner, J., Kawohl, T., Kreft, H., Soria-Auza,
R.W., Zimmermann, N.E., Linder, H.P., Kessler, M., 2017. Climatologies
at high resolution for the earth's land surface areas. Sci Data 4,
170122. \url{https://doi.org/10.1038/sdata.2017.122}

\bibitem[\citeproctext]{ref-KopeckyCizkova2010}
Kopecký, M., Čížková, Š., 2010. Using topographic wetness index in
vegetation ecology: Does the algorithm matter? Appl. Veg. Sci. 13,
450--459. \url{https://doi.org/10.1111/j.1654-109X.2010.01083.x}

\bibitem[\citeproctext]{ref-LeempoelEtAl2015}
Leempoel, K., Parisod, C., Geiser, C., Daprà, L., Vittoz, P., Joost, S.,
2015. Very high-resolution digital elevation models: Are multi-scale
derived variables ecologically relevant? Methods in Ecology and
Evolution 6, 1373--1383. \url{https://doi.org/10.1111/2041-210X.12427}

\bibitem[\citeproctext]{ref-Lindsay2016}
Lindsay, J.B., 2016. Whitebox {GAT}: {A} case study in geomorphometric
analysis. Comput. Geosci. 95, 75--84.
\url{https://doi.org/10.1016/j.cageo.2016.07.003}

\bibitem[\citeproctext]{ref-LiuEtAl2013}
Liu, C., White, M., Newell, G., 2013. Selecting thresholds for the
prediction of species occurrence with presence-only data. Journal of
Biogeography 40, 778--789. \url{https://doi.org/10.1111/jbi.12058}

\bibitem[\citeproctext]{ref-MartinEtAl2016}
Martin, L., Edwards, Y.H., Roe, J., Garrard, A., 2016. Faunal turnover
in the {Azraq Basin}, eastern {Jordan} 28,000 to 9,000 cal {BP},
signalling climate change and human impact. Quaternary Research 86,
200--219. \url{https://doi.org/10.1016/j.yqres.2016.07.001}

\bibitem[\citeproctext]{ref-MillerEtAl2024}
Miller, T., Blackwood, C.B., Case, A.L., 2024. Assessing the utility of
{SoilGrids250} for biogeographic inference of plant populations. Ecology
and Evolution 14, e10986. \url{https://doi.org/10.1002/ece3.10986}

\bibitem[\citeproctext]{ref-ModEtAl2016}
Mod, H.K., Scherrer, D., Luoto, M., Guisan, A., 2016. What we use is not
what we know: Environmental predictors in plant distribution models.
Journal of Vegetation Science 27, 1308--1322.
\url{https://doi.org/10.1111/jvs.12444}

\bibitem[\citeproctext]{ref-StockwellPeterson2002}
Stockwell, D.R.B., Peterson, A.T., 2002. Effects of sample size on
accuracy of species distribution models. Ecological Modelling 148,
1--13. \url{https://doi.org/10.1016/S0304-3800(01)00388-X}

\bibitem[\citeproctext]{ref-ValaviEtAl2021}
Valavi, R., Elith, J., Lahoz-Monfort, J.J., Guillera-Arroita, G., 2021.
Modelling species presence-only data with random forests. Ecography 44,
1731--1742. \url{https://doi.org/10.1111/ecog.05615}

\bibitem[\citeproctext]{ref-ValaviEtAl2022}
Valavi, R., Guillera-Arroita, G., Lahoz-Monfort, J.J., Elith, J., 2022.
Predictive performance of presence-only species distribution models: A
benchmark study with reproducible code. Ecological Monographs 92,
e01486. \url{https://doi.org/10.1002/ecm.1486}

\bibitem[\citeproctext]{ref-VelazcoEtAl2017}
Velazco, S.J.E., Galvão, F., Villalobos, F., De Marco Júnior, P., 2017.
Using worldwide edaphic data to model plant species niches: {An}
assessment at a continental extent. PLoS One 12, e0186025.
\url{https://doi.org/10.1371/journal.pone.0186025}

\bibitem[\citeproctext]{ref-WiszEtAl2008}
Wisz, M.S., Hijmans, R.J., Li, J., Peterson, A.T., Graham, C.H., Guisan,
A., Group, N.P.S.D.W., 2008. Effects of sample size on the performance
of species distribution models. Diversity and Distributions 14,
763--773. \url{https://doi.org/10.1111/j.1472-4642.2008.00482.x}

\bibitem[\citeproctext]{ref-WrightZiegler2017}
Wright, M.N., Ziegler, A., 2017. Ranger: {A Fast Implementation} of
{Random Forests} for {High Dimensional Data} in {C}++ and {R}. Journal
of Statistical Software 77, 1--17.
\url{https://doi.org/10.18637/jss.v077.i01}

\bibitem[\citeproctext]{ref-YaworskyEtAl2023}
Yaworsky, P.M., Hussain, S.T., Riede, F., 2023. Climate-driven habitat
shifts of high-ranked prey species structure {Late Upper Paleolithic}
hunting. Scientific Reports 13, 4238.
\url{https://doi.org/10.1038/s41598-023-31085-x}

\end{CSLReferences}




\end{document}
